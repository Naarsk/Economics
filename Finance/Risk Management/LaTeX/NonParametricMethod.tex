%! Author = Leonardo Cruciani
%! Date = 29/03/2025

% Preamble
\documentclass[12pt]{article}

% Packages
\usepackage[top=1in, bottom=1.2in, left=1in, right=1in]{geometry}
\usepackage{amsmath}
\usepackage{upgreek}
\usepackage{graphicx}
\usepackage{wrapfig}
\usepackage{hyperref}
\usepackage{tikzexternal}
\setlength{\parindent}{0pt}

\author{Leonardo Cruciani}
\title{Forecasting VaR using Garch and KDE}
% Document
\begin{document}
    \section{Introduction}
    \section{Methodology}
    \subsection{Data}
        The dataset contains the closing prices of some popular indices (S\&P500, FTSE, Eurostoxx50 and Euronext100) from 2006 to 2024.
        The dataset is split in two, with the first 11 years being used for training and the last 7 years for testing.
        From the dataset the log returns are computed, and aggregated for 5, 10 and 20 days.

    \subsection{Garch}
        Since its inception, various Garch models have been used in the field of finance.
        In the paper \cite{thorsten} an augmented Garch model that nests all the most commonly used (e.g. Garch(1,1), GJR-Garch, Aparch, etc) is proposed.
        In this work I will resort to a simpler Power Garch model with leverage effect to model the volatility $\sigma_t$:
        \begin{equation}
            \sigma_t^\kappa = \omega
            + \alpha \left|\epsilon_{t-1}\right|^\kappa
            + \beta \sigma_{t-1}^\kappa
            + \gamma \left|\epsilon_{t-1}\right|^\kappa I_{[\epsilon_{t-1}<0]}
        \end{equation}
        Where $\epsilon_t \sim D(0,1)$ are independent shocks distributed according to a \textit{standardized distribution} $D(0,1)$. In the paper \cite{thorsten} a
        Truncated Skewed Levy distribution, is used for the shocks, but I instead will use the kernel density estimated distribution. More about this in the next section.\\
        The parameters to be estimated are: $\alpha$, which models the sensitivity of the volatility to the shock, $\beta$, which models the persistence of the volatility, $\gamma$, which models the leverage effect, the exponent $\kappa$ (typically $\simeq 2$) and the so called (when $\kappa=2$) long term variance $\omega$.\\
        To model the return $r_t$ I used the following model:
        \begin{equation}
                r_t = \mu + \sigma_t \epsilon_t,
        \end{equation}
        Where the mean $\mu$ can be safely assumed to be constant for short term returns.
        The parameters are calibrated on historical data and the model is then used to forecast the next day variance as returns become available.
    \subsection{Kernel Density Estimation}
        The closing prices in the dataset are standardized following the formula:
        \begin{equation}
            z_t = \frac{r_t - \mu}{\sigma_t}
        \end{equation}
        and a kernel density estimator is used to estimate the probability distribution of the standardized returns.
        I tested two different ways to determine the bandwith parameter of the estimator:
        \begin{itemize}
            \item by Scott’s rule of thumb
            \item by picking the value that minimizes the discrepancy between $\alpha$, and the percentage of violations on the training set of the $\alpha$\% Value at Risk calulated form the KDE-estimated distribution.
        \end{itemize}

    \subsection{Value at Risk Violations}
        On the testing set, the $\alpha$\% Value at Risk is calulated by multiplying the $\alpha$-percentile of the shock distribution with the estimated volatility (using the model calibrated on the training set).
        The estimated VaR is then compared with the actual returns, and a violation is flagged if the return is below the (negative) VaR.
\end{document}